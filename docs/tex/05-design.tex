\chapter{Конструкторская часть}

\section{Требования к программному обеспечению}

Разрабатываемое программное обеспечение должно предоставлять следующие
функциональности:
\begin{itemize}
    \item изменение параметров одежды (массы узлов, жесткости пружин, цвета,
        коэффициентов рассеянного и зеркального отражения) в интерактивном
        режиме;
    \item изменение параметров окружающей среды (гравитации, ветра);
    \item вращение, перемещение и масштабирование камеры.
\end{itemize}

При этом программа должна удовлетворять следующим требованиям:
\begin{itemize}
    \item время отклика программы должно быть не более 1 секунды для корректной
        работы в интерактивном режиме;
    \item программа должна корректно реагировать на любые действия пользователя.
\end{itemize}

\section{Разработка алгоритмов}

В данном подразделе представлены схемы выбранных алгоритмов.

\subsection{Алгоритм моделирования ткани}

На рисунке \ref{img:massPos} представлена схема алгоритма вычисления нового
положения точечной массы. На рисунке \ref{img:force} представлена схема
алгоритма вычисления результирующей силы, действующей на точечную массу.

\img{15cm}{massPos}{Схема алгоритма вычисления нового положения точечной
массы}{massPos}

\img{23cm}{force}{Схема алгоритма вычисления результирующей силы, действующей на
точечную массу}{force}

\clearpage
\subsection{Алгоритм синтеза изображения}

На рисунке \ref{img:opengl} представлена схема алгоритма синтеза
изображения с помощью выбранного API OpenGL.

\img{20cm}{opengl}{Схема алгоритма синтеза изображения}{opengl}

\section{Описание структуры программы}

На рисунке \ref{img:struct} представлена диаграмма классов разрабатываемого
программного обесепечения.

\img{17cm}{classes}{Диаграмма классов}{struct}

В разрабатываемом программном обеспечении реализуются следующие классы:
\begin{itemize}
    \item MyGL -- класс-виджет для отрисовки сцены;
    \item Shader --- класс подключения шейдров;
    \item Object --- базовый класс объектов сцены;
    \item Camera --- класс камеры;
    \item MassSpringModel --- класс массо-пружинной модели;
    \item Mass --- класс точечной массы;
    \item Spring --- класс пружины;
    \item Pattern --- базовый класс тканных изделий;
    \item Cloth --- класс прямоугольной ткани;
    \item TShirt --- класс футболки.
\end{itemize}

\section{Используемые типы и структуры данных}

При реализации программного обеспечения используются следующие типы и
структуры данных:
\begin{itemize}
    \item
\end{itemize}




