\chapter{Аналитическая часть}

\section{Методы визуализации одежды}

Как уже было сказано во введении, одежда является более сложной формой ткани,
поэтому далее будут рассмотрены методы моделирования тканных материалов.
Данные методы можно разделить на два основных типа:
\begin{itemize}[left=\parindent]
    \item геометрические методы;
    \item физические методы.
\end{itemize}

\subsection{Геометрические методы}

Геометрические методы не учитывают физические свойства ткани, они фокусируются
на воспроизведении внешнего вида тканных материалов с помощью представления
поверхности математическими функциями. Таким образом, в данных методах не
требуется решение сложных систем уравнений, что дает им преимущество в виде
большой скорости выполнения \cite{bib07}.

Хотя геометрические методы за короткое время могут с достаточной долей
реалистичности визуализировать ткань, каждый из них либо решает достаточно
специфическую задачу, например, воспроизведение висящей ткани или моделирование
складок на рукаве, либо нуждается в активном содействии пользователя, что
уменьшает количество сфер, в которых их можно применить \cite{bib07}.

Геометрическими методами являются:
\begin{itemize}[left=\parindent]
    \item метод моделирования свисающей ткани \cite{bib08};
    \item метод моделирования складок на рукаве \cite{bib07};
    \item методы со значительной степенью вмешательства пользователя
        \cite{bib09}\cite{bib10}.
\end{itemize}

\subsection{Физические методы}

В физических методах модель ткани представляют в виде  треугольных или
прямоугольных сеток с точечными массами в узлах. Взаимодействие между этими
массами  описываются различными способами в зависимости от метода. В моделях,
основанных на энергии, положение точки определяется энергетическим состоянием
системы, а именно: ищется такое состояние ткани, в котором энергия системы
минимальна. В других моделях силы взаимодействия между точечными массами
описываются дифференциальными уравнениями, решение которых производится с
помощью численного интегрирования, в результате чего получают координаты точки.
\cite{bib07}

Так как в физических методах проводится большое количество вычислений: решение
системы дифференциальных уравнений или перебор состояний системы для поиска
минимумов энергии, --- скорость их выполнения ниже, чем у геометрических
методов. Однако физические методы предоставляют большую свободу: мы можем
создать реалистичное изображение без привлечения пользователя, а также можем
смоделировать разные виды ткани, изменяя физические характеристики (например,
увеличение значения массы в узлах приведет к утяжелению ткани), что также
позволяет сделать модель более правдоподобной. \cite{bib07}

Основными физическими моделями являются \cite{bib11}:
\begin{itemize}[left=\parindent]
    \item модель сплошной среды (Continuum Model) \cite{bib12};
    \item энергетическая модель систем частиц (Energy-Based Particle Systems Model) \cite{bib13};
    \item массо-пружинная модель (Mass-Spring Model) \cite{bib14}.
\end{itemize}

\section{Методы разрешения пересечений и самопересечений}

Предполагается пересечение с торсом + складки, поэтому надо добавить, возможно
впишется в предыдущий раздел
Может быть, надо где-то описать методы соединения деталей одежды

\section{Существующие программные обеспечения}

Готовое ПО + какие методы использованы

\section{Модель представления одежды}

Mass-Spring Model 
Подробное описание выбранного метода
