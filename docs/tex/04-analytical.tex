\chapter{Аналитическая часть}

\section{Одежда, как объект физического мира}

Любая одежда: футболки, брюки, носки, шарфы и куртки и т.~д., --- представляет
собой одну или несколько соединенных между собой деталей из ткани. Ткань, в свою
очередь, состоит из натуральных или искусственных волокон или нитей, которые
производятся путем прядения различных материалов, таких как шерсть, хлопок,
лен~и~т.~п. Для соединения полученных волокон используют следующие техники:
\begin{itemize}[left=\parindent]
    \item ткачество --- изготовление ткани путем переплетения нитей под прямым
        углом (рисунок \ref{img:wrap});
    \item вязание --- соединение волокон между собой путем образования и
        протягивания петель (рисунок \ref{img:knit});
    \item макраме --- закрепление нитей с помощью узлов;
    \item получение войлока --- прессование волокон животных.
\end{itemize}
После применения одной из этих техник получается готовая ткань.

\twoimg{40mm}{img01}{Строение тканного
полотна~\cite{imgSite01}}{wrap}{img02}{Строение вязаного
полотна~\cite{imgSite01}}{knit}

Разные волоконные материалы и техники их соединения являются причиной различной
степени проявления разными тканями основных механических свойств: растяжения,
сдвига и изгиба. Отсутствие у тканных материалов упругих свойств приводит к
образованию складок и легком драпировании на другие объекты. Таким образом,
разное строение ткани в зависимости от типа волокон, большое количество узлов,
из которых она состоит, разные степени проявления механических свойств, большое
количество степеней свободы -- все это следует учитывать при моделировании
ткани. Вследствие отсутствия возможности учесть все необходимые параметры ткани
разрабатываются упрощенные модели и методы их реализации для решения конкретных
задач: в одних целью является максимальная реалистичность изображения, в других
--- высокая скорость работы с наименьшей потерей реалистичности~\cite{bib11}.

\section{Методы визуализации одежды}

Как уже было сказано выше, одежда является более сложной формой ткани,
поэтому далее будут рассмотрены методы моделирования тканных материалов.
Данные методы можно разделить на два основных типа:
\begin{itemize}[left=\parindent]
    \item геометрические методы;
    \item физические методы.
\end{itemize}

\subsection{Геометрические методы}

Геометрические методы~\cite{bib07} не учитывают физические свойства ткани, они
фокусируются на воспроизведении внешнего вида тканных материалов с помощью
представления поверхности математическими функциями. Таким образом, в данных
методах не требуется решение сложных систем уравнений, что дает им преимущество
в виде большой скорости выполнения.

Хотя геометрические методы за короткое время могут с достаточной долей
реалистичности визуализировать ткань, каждый из них либо решает достаточно
специфическую задачу, например, воспроизведение висящей ткани или моделирование
складок на рукаве, либо нуждается в активном содействии пользователя, что
уменьшает количество сфер, в которых их можно применить.

\subsubsection{Метод моделирования свисающей ткани}
Метод моделирования свисающей ткани~\cite{bib08} предназначен для моделирования
тканного материала, который закреплен на некотором количестве точек. Ткань
считается прямоугольной и представляется в виде сетки, а моделирование
выполняется в два этапа:
\begin{enumerate}[label=\arabic*)]
    \item На первом этапе каждая пара заданных точек соединяется цепной кривой,
        представленной формулой~(\ref{eq:01}):

        \begin{equation}\label{eq:01}
            y = a\cosh(\frac{x}{a}),
        \end{equation}

        где $a$ --- коэффициент масштабирования.

        Если при этом проекции двух кривых на плоскость $XOZ$ пересекаются как
        показано на рисунке \ref{img:Weil}, то для устранения дополнительных
        вычислений, нижняя кривая удаляется. После завершения этого этапа
        модель представляет изогнутый каркас, форма которого только приближается
        к структуре ткани. 

        \img{5cm}{img03}{Две пересекающиеся цепные кривые}{Weil}

    \item Для получения более точной модели добавляются новые поверхности,
        которые создаются путем разделения треугольников, образованных цепными
        кривыми, вычисленных на предыдущем этапе.  Новые треугольники также
        разделяются. Итерационный процесс продолжается до тех пор, пока
        максимальное смещение точек сетки за один проход не станет меньше
        заданного значения.
\end{enumerate}

После завершения двух этапов полученная модель визуализируется с
предварительным нанесением на нее сплайновых кривых для получения
гладкого изображения ткани~\cite{bib07, bib08}.

\subsubsection{Метод моделирования складок на рукаве}

Метод моделирования складок на рукаве~\cite{bib07} ориентирован под конкретную
задачу, а именно моделирование рукава на сгибающейся руке.  Ткань представляется
в виде полого цилиндра, состоящего из набора окружностей~$R_i$~(рисунок
\ref{img:Agui01}). Складки образуются в том случае, если модуль разности
расстояний между точками двух соседних окружностей до ($L_0$) и после
($L_{i,j}$) деформации (рисунок \ref{img:Agui02})
меньше некоторого заранее заданного порогового значения. Складки моделируются
путем преобразования окружностей в многоугольники и изменения положения их
вершин~\cite{bib07}.

\img{5cm}{img04}{Модель рукава: полый цилиндр из набора окружностей}{Agui01}
\img{5cm}{img05}{Одна секция рукава до и после деформации}{Agui02}

\subsubsection{Методы со значительной степенью вмешательства пользователя}

Методы со значительной степенью вмешательства пользователя~\cite{bib09, bib10}
разрабатывались специально под графические редакторы. Основная их идея состоит в
том, чтобы изначально представить одежду как полностью прилегающую к телу или
расположенную на небольшом расстоянии с повторением контуров тела, а далее
предоставить пользователю интерфейс для редактирования положения ткани.  Так как
в данных методах реалистичность итогового изображения во многом зависит от
пользователя, а в этой работе ставится задача получения реалистичного
изображения без его вмешательства, более подробное описание этих методов здесь
не приводится.

\subsection{Физические методы}

В физических методах~\cite{bib07} модель ткани представляют в виде треугольных
или прямоугольных сеток с точечными массами в узлах. Взаимодействие между этими
массами  описываются различными способами в зависимости от метода. В моделях,
основанных на энергии, положение точки определяется энергетическим состоянием
системы, а именно: ищется такое состояние ткани, в котором энергия системы
минимальна. В других моделях силы взаимодействия между точечными массами
описываются дифференциальными уравнениями, решение которых производится с
помощью численного интегрирования, в результате чего получают координаты точки.

Так как в физических методах проводится большое количество вычислений: решение
системы дифференциальных уравнений или перебор состояний системы для поиска
минимумов энергии, --- скорость их выполнения ниже, чем у геометрических
методов. Однако физические методы предоставляют большую свободу: мы можем
создать реалистичное изображение без привлечения пользователя, а также можем
смоделировать разные виды ткани, изменяя физические характеристики (например,
увеличение значения массы в узлах приведет к утяжелению ткани), что также
позволяет сделать модель более правдоподобной.

\subsubsection{Модель сплошной среды}

В модели сплошной среды ткань рассматривается в виде сплошной, однородной
структуры. Ее поведение моделируется с помощью теории упругости, из физически
обоснованных выражений которой выводится большая система обыкновенных
дифференциальных уравнений. Решение этой системы производится численно. Такой
подход позволяет модели естественным образом реагировать на приложенные силы,
окружающую среду и другие объекты, однако требует дорогих вычислительных
затрат~\cite{bib12, bib15}.

\subsubsection{Энергетическая модель системы частиц}

В методе энергетической модели системы частиц~\cite{bib13} точки пересечения
нитей ткани рассматриваются как точечные массы или частицы, а моделирование
состоит из двух этапов. На первом этапе учитывается гравитация и  определяются
любые столкновения с объектами или землей. Позиции частиц определяются с помощью
уравнения, описывающегося формулой~(\ref{eq:02}):

\begin{equation}\label{eq:02}
    m\vec{a}+c\vec{v}=m\vec{g},
\end{equation}

где $m$ --- масса частицы,

~~~~~~$\vec{a}$ --- вектор ускорения,

~~~~~~$c$ --- сопротивление воздуха,

~~~~~~$\vec{v}$ --- вектор скорости,

~~~~~~$\vec{g}$ --- вектор ускорения свободного падения.

После первого этапа получается грубая модель ткани. Для получения реалистичного
изображения вводят второй этап, на котором минимизируется энергия системы
частиц, где энергия каждой частицы $U_i$ представляется, как сумма энергий
основных взаимодействий отталкивания $U_{repel_i}$, растяжения $U_{stretch_i}$,
сдвига $U_{shear_i}$, изгиба $U_{bend_i}$ и гравитации
$U_{gravity_i}$, что описано формулой~(\ref{eq:03})~\cite{bib07}:

\begin{equation}\label{eq:03}
U_i=U_{repel_i}+U_{stretch_i}+U_{shear_i}+U_{bend_i}+U_{gravity_i}.
\end{equation}

С помощью этого метода достигается реалистичность итогового изображения, однако
для поиска минимума энергии требуются большие временные затраты~\cite{bib11}.

\subsubsection{Массо-пружинная модель}

В массо-пружинной модели ткань также представляется в виде сетки с точечными
массами в узлах, но частицы между собой связаны пружинами, которые отвечают за
упругое поведение материала (рисунок~\ref{img:Provot}). Такое представления
ткани практично, так как криволинейные поверхности часто представляются
полигональными сетками, вершины которых можно рассматривать, как точечные массы,
а ребра --- как пружины~\cite{bib15}. Каждая пружина в этой модели относится к
одному из типов~\cite{bib14}:
\begin{itemize}[left=\parindent]
    \item пружины структуры ткани соединяют частицы с индексами $[i,j]$
        и $[i+1,j]$, а также $[i,j]$ и $[i, j+1]$;
    \item пружины сдвига ткани соединяют частицы с индексами $[i,j]$ и
        $[i+1,j+1]$, а также $[i+1,j]$ и $[i, j+1]$;
    \item пружины изгиба ткани соединяют частицы с индексами $[i,j]$ и
        $[i+2,j]$, а также $[i,j]$ и $[i, j+2]$.
\end{itemize}

\img{5cm}{img06}{Сетка масс и пружин}{Provot}

После представления ткани в виде сетки для каждой точки вычисляется
результирующая сила, вычисляемая путем сложения внутренних и внешних сил, а
положение точки определяется путем явного интегрирования полученных
уравнений~\cite{bib11}. Так как данный метод предназначен для анимации ткани,
скорость вычислений в нем выше, чем в ранее описанных методах~\cite{bib14}.

\subsection*{Вывод}

На основе поставленных целей и задач для визуализации одежды была выбраны
критерии оценки описанных методов (универсальность, необходимость вмешательства
пользователя и скорость вычислений) и составлена таблица~\ref{tab:cloth}
сравнения, после анализа которой выбрана массо-пружинная модель ткани. Этот
метод, как физический, имеет преимущество перед геометрическими, так как он
универсален, а реалистичность итогового изображения не зависит от действий
пользователя. Среди физических эта модель выделяется скоростью работы, которая
требуется для создания динамического изображения и взаимодействия с
пользователем. Также эта модель является интуитивно понятной в силу базовых
физических законов, лежащий в ее основе, что является преимуществом как для
реализации, так и для построения понятного для пользователя интерфейса.

\noindent
\begin{longtable}[Hc]{|p{4cm}|p{2cm}|p{4cm}|p{2cm}|}
\captionsetup{format=hang,justification=raggedright,
              singlelinecheck=off,width=16.3cm}
\caption{Сравнение методов моделирования ткани\label{tab:cloth}}\\
    \hline
    \multicolumn{1}{|c|}{\textbf{Метод}} &
    \multicolumn{1}{c|}{\textbf{Универсальность}} &
    \multicolumn{1}{p{4cm}|}{\textbf{Вмешательство пользователя}} &
    \multicolumn{1}{c|}{\textbf{Скорость}}\\
    \hline
    Свисающей ткани & - & не требуется & высокая \\
    \hline
    Складок на рукаве & - & не требуется & высокая \\
    \hline
    С вмешательством пользователя & + & требуется & низкая \\
    \hline
    Сплошной среды & + & не требуется & низкая \\
    \hline
    Энергетический & + & не требуется & низкая \\
    \hline
    Массо-пружинная модель & + & не требуется & средняя \\
    \hline
\end{longtable}

\section{Описание реализуемого метода}

В качестве реализуемого метода была выбрана массо-пружинная модель~\cite{bib14},
в которой ткань представляется в виде сетки точечных масс, соединенных пружинами
так, как показано на рисунке~\ref{img:msm}.

\img{9cm}{img07}{Пружины точечной массы}{msm}

Моделирование ткани происходит с шагом по времени $\Delta t$. На каждом шаге
определяется новое положение каждой точечной массы на основе предыдущего
положения точки и шага $\Delta t$.

На каждую точечную массу действуют внутренние и
внешние:
\begin{itemize}
    \item сила упругости со стороны каждой из связанных с данной точечной массой
        пружин, определяющаяся формулой~(\ref{eq:04}):
        \begin{equation}\label{eq:04}
            \vec{F}_{\text{упр}} = k \cdot [\vec{l} -
            l_0\frac{\vec{l}}{||\vec{l}||}],
        \end{equation}

        где $k$ --- коэффициент жесткости пружины;

        ~~~~~$\vec{l}$ --- вектор от данной точечной массы к той, с которой ее
        соединяет пружины (его модуль определяет текущую длину пружины);

        ~~~~~$l_0$ --- длина пружины в недеформированном состоянии.

    \item сила сопротивления воздуха, вычисляемая по формуле~(\ref{eq:05}):
        \begin{equation}\label{eq:05}
            \vec{F}_{\text{сопр}} = - C_{\text{сопр}}\vec{v},
        \end{equation}

        где $C_{\text{сопр}}$ --- коэффициент сопротивления;
        
        ~~~~~$\vec{v}$ --- текущая скорость точечной массы.

    \item сила тяжести (формула~(\ref{eq:06})):
        \begin{equation}\label{eq:06}
            \vec{F}_{\text{тяж}} = m\vec{g},
        \end{equation}

        где $m$ --- масса точки;

        ~~~~~$\vec{g}$ --- ускорение свободного падения.

    \item сила ветра, каждая из трех составляющих вектора которой
        представляется комбинацией тригонометрических функций,
        зависящих от координат точечной массы и времени.
\end{itemize}

Результирующая сила $\vec{F}$, действующая на точечную массу, определена как
векторная сумма всех действующих сил.

По второму закону Ньютона определяет ускорение данной точки
(формула~(\ref{eq:08})), которая является второй производной координаты по
времени. 
\begin{equation}\label{eq:08}
    \vec{a} = \frac{\vec{F}}{m},
\end{equation}

        где $\vec{a}$ --- ускорение точечной массы;

        ~~~~~$m$ --- масса точки.

Новое положение точки определяется путем численного интегрирования ускорения.  В
качестве метода численного интегрирования может использоваться метод Эйлера в
данном случае представляющий уравнения движения, описывающиеся
формулой~(\ref{eq:09}). 
\begin{equation}\label{eq:09}
    \begin{cases}
        \vec{a}_{t+\Delta t} = \frac{\vec{F}_{t}}{m} \\
        \vec{v}_{t+\Delta t} = \vec{v}_{t} + \Delta t\vec{a}_{t+\Delta t} \\
        \vec{P}_{t+\Delta t} = \vec{P}_{t} + \Delta t\vec{v}_{t+\Delta t} \\
    \end{cases}
    ,
\end{equation}

где $\vec{a}_{t+\Delta t}, \vec{v}_{t+\Delta t}, \vec{P}_{t+\Delta t}$ ---
ускорение, скорость и радиус-вектор точечной массы на текущем шаге;

~~~~~$\vec{F}_{t}, \vec{v}_{t}, \vec{P}_{t}$ --- сила, скорость и радиус-вектор
точечной массы на предыдущем шаге.

Для повышения стабильности системы в данной работе используется метод численного
интегрирования Верле~\cite{bib01}. Получение формулы для определения нового
положения методом Верле из уравнения движения определяется последовательностью
преобразований, описывающихся формулами~(\ref{eq:10})-(\ref{eq:12}).  Таким
образом, значение координаты точечной массы на каждом шаге определяется по
формуле~(\ref{eq:13}).
\begin{equation}\label{eq:10}
    \vec{P}_{t+\Delta t} = \vec{P}_{t} + \Delta t\vec{v}_{t} +
    \frac{\vec{a}_{t+\Delta t}\Delta{t}^2}{2}, \\
\end{equation}

\begin{equation}\label{eq:11}
    \vec{v}_{t} = \frac{\vec{P}_{t} - \vec{P}_{t-\Delta t}}{\Delta t},
\end{equation}

где $\vec{P}_{t-\Delta t}$ --- радиус-вектор точечной массы на два шага ранее.

\begin{equation}\label{eq:12}
    \vec{P}_{t+\Delta t} = \vec{P}_{t} + \vec{P}_{t} - \vec{P}_{t-\Delta t} +
    \frac{\vec{a}_{t+\Delta t}\Delta{t}^2}{2}, \\
\end{equation}

\begin{equation}\label{eq:13}
    \vec{P}_{t+\Delta t} = 2\vec{P}_{t} - \vec{P}_{t-\Delta t} +
    \frac{\vec{a}_{t+\Delta t}\Delta{t}^2}{2}. \\
\end{equation}

\section{Формализация модели}

Основным элементом сцены является футболка, которая, как и в реальной
жизни, состоит из нескольких частей --- выкроек. Для упрощения моделирования
футболка представляется в виде двух одинаковых частей, каждая из которых
является сеткой из масс и пружин. В силу возможности первоначального
расположения масс по квадратной сетки, форма выкроек задается матрицей, в
которой единица соответствует наличию массы, ноль --- ее отсутствию. Соединение
частей происходит с помощью введения дополнительных пружин у частиц
расположенных по местам швов.

Свойства тканного материала могут быть описаны массой узлов сетки и жесткостью
пружин, а также коэффициентами рассеянного и зеркального отражения света
от точечного источника света.

Таким образом, модель футболки описывается следующими параметрами:
\begin{itemize}
    \item форма частей --- матрица из нулей и единиц;
    \item масса узла сетки $m$ --- вещественное число от 0.001 до 1 (граммы);
    \item жесткость пружины $k$ --- целое число от 100 до 25000 (ньютон/метр);
    \item коэффициент рассеянного отражения --- вещественное число от 0 до 1;
    \item коэффициент зеркального отражения --- вещественное число от 0 до 1.
\end{itemize}

\section{Методы рендеринга изображения}

Рендеринг (rendering, визуализация)~\cite{bib16} --- в компьютерной графике,
процесс преобразования цифровых моделей в визуализируемое представление ---
изображение.

Реализуемым методом моделирования одежды выбрана массо-пружинная модель, в
которой используется большое количество узлов.
Отрисовка модели одежды должна происходить с такой скоростью, чтобы обеспечить
непрерывное движение, а также взаимодействие с пользователем.
Стандартной графической библиотеки с программной реализацией алгоритмов
отрисовки не сможет обеспечить необходимую скорость работы, поэтому необходимо
выбрать прикладной программный интерфейс (API), который позволит получить
необходимую скорость синтеза изображения. Далее рассматриваются такие API,
как OpenGL, Vulkan и DirectX.

\subsection{OpenGL}

OpenGL (Open Graphics Library --- «Открытая Графическая
Библиотека»)~\cite{bib17} --- это прикладной программный интерфейс для
разработки приложений в области двумерной и трехмерной графики. OpenGL имеет
продуманную внутреннюю структуру и процедурный интерфейс, с помощью которых
можно создавать сложные и мощные программные комплексы, затрачивая при этом
минимальное время по сравнению с другими графическими библиотеками. Рендеринг
изображения базируется на конвейерной обработке, то есть графические данные
прежде, чем стать конечным изображением, проходят несколько этапов обработки.

Преимущества:
\begin{itemize}
    \item кроссплатформенность;
    \item открытый код;
    \item высокая производительность;
    \item поддержка многими языками программирования и аппаратными
        графическими устройствами.
\end{itemize}

Недостатки:
\begin{itemize}
    \item сложность работы с новыми поколениями GPU;
    \item отсутствие обработки устройств ввода-вывода и поддержки менеджера
        окон.
\end{itemize}

\subsection{Vulkan}

Vulkan~\cite{bib17} --- это первый открытый низкоуровневый API, который
предоставляет разработчикам прямой доступ к GPU для полного контроля над его
работой. Как и OpenGL, Vulkan --- это бесплатный стандарт с открытым кодом,
доступный для любой платформы.

Преимущества:
\begin{itemize}
    \item кроссплатформенность;
    \item открытый код;
    \item высокая производительность.
\end{itemize}

Недостатки:
\begin{itemize}
    \item работа напрямую с GPU (сложно для неподготовленного программиста);
    \item отсутствие поддержка многими языками программирования.
\end{itemize}

\subsection{DirectX}

DirectX~\cite{bib17} --- это набор API, разработанных для решения задач
компьютерной графики под управлением Microsoft Windows. Наиболее широко
используется при написании компьютерных игр.

Преимущества:
\begin{itemize}
    \item высокая производительность;
    \item высокое качество изображения.
\end{itemize}

Недостатки:
\begin{itemize}
    \item работа только под управлением Microsoft Windows;
    \item отсутствие поддержки старого оборудования новыми версиями. 
\end{itemize}

\subsection*{Вывод}

С учетом представленных описаний программных интерфейсов выбраны критерии их
оценки (кроссплатформенность, открытый код, поддержка языком Python) и
составлена таблица~\ref{tab:api} сравнение, после анализа которой для
визуализации модели одежды был выбран OpenGL, он является кроссплатформенным,
что дает ему преимущество перед DirectX, и поддерживается многими языками
программирования, что дает ему преимущество перед Vulkan.

\noindent
\begin{longtable}[Hc]{|c|c|c|c|}
\captionsetup{format=hang,justification=raggedright,
              singlelinecheck=off,width=16.3cm}
\caption{Сравнение графических API\label{tab:api}}\\
    \hline
    \multicolumn{1}{|c|}{\textbf{API}} &
    \multicolumn{1}{c|}{\textbf{Кроссплатформенность}} &
    \multicolumn{1}{c|}{\textbf{Открытый код}} &
    \multicolumn{1}{c|}{\textbf{Поддержка Python}}\\
    \hline
    OpenGL    & + & + & + \\
    \hline
    Vulkan    & + & + & - \\
    \hline
    DirectX   & - & - & + \\
    \hline
\end{longtable}

\section{Существующие программные обеспечения}

Реалистичная визуализация одежды является востребованной в современном мире.
Так, она используется в профессиональных коммерческих программных обеспечениях
для индустрии моды, таких как Clo~\cite{site01} (рисунок \ref{img:Clo}),
Browzwear~\cite{site02} (рисунок \ref{img:Browzwear}).

\twoimg{50mm}{img13}{Пример работы программы Clo}{Clo}{img14}{Пример работы
программы Browzwear}{Browzwear}

Не являются исключением и бесплатные кроссплатформенные приложения для работы с
3D графикой, таких как Blender. Данное приложение предоставляет большой
функционал для моделирования, симуляции, рендеринга, монтажа, записи видео и
создания игр. Дополнения к Blender упрощают моделирование тех, или иных
объектов. В том числе существует дополнение для моделирования ткани и одежды,
пример которого представлен на рисунке \ref{img:blender}.

\img{6cm}{img15.jpg}{Пример визуализации одежды в Blender}{blender}

Также существует аппаратная реализация визуализации одежды PhysX
Clothing~\cite{site03}, пример работы которой представлен на рисунке
\ref{img:physix}.

\img{6cm}{img16.jpg}{Пример визуализации одежды PhysX Clothing}{physix}

\section*{Вывод}

В данном разделе были описаны методы визуализации ткани, как части
одежды, проанализированы методы рендеринга изображения, формально описана
модель футболки.

В качестве реализуемого метода выбрана и описана массо-пружинная модуль, в
качестве методы рендеринга изображения выбран OpenGL.
