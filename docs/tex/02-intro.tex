\chapter*{Введение}
\addcontentsline{toc}{chapter}{Введение}

Современные исследования в области компьютерной графики сосредоточены на
моделировании и визуализации явлений реального мира с максимальной
реалистичностью. Моделирование одежды и, как более общего случая, ткани играют
не последнюю роль в детализации виртуальных сред.\cite{Simnett2012RealtimeSA}
Реалистичный вид одежды придает выразительности анимационным персонажам в
комьютерных играх и мультипликации \cite{Zurdo2013AnimatingWB}; в фильмах
помогает сделать неотличимыми реального человека, снятого на камеру, от, так
называемого, цифрового дублера ~--- виртуальной реалистичной копии, которая
"выполняет"\ сложные,  опасные для жизни трюки. \cite{Stuyck2018ClothSF} Также
сегодня развивается идея виртуальной примерочной в интернет-магазинах
\cite{Keckeisen2005PhysicalCS}. Все это показывает практическую применимость
моделирования одежды, а следовательно, необходимость разработки методов её
визуализации.

Тканные материалы обладают уникальными свойствами: гибкая, эластичная, легко
изменяет свою форму, хаотична (зависит от предыдущих положений -> выглядит
по-разному) Все это создает сложности при моделировании одежды. Полужило
поводом к разработки разнообразных методов, предназначенных для различных
целей.

Цель работы ~--- разработать программное обеспечение для реалистичной
визуализации плечевой одежды на примере футболки, предоставляющее возможность
изменения её положения (перемещение, вращение, масштабирование).

Для достижения поставленной цели необходимо решить следующие задачи:
\begin{itemize}[left=\parindent] \item  формально описать модель ткани, как
            части одежды; \item  проанализировать методы визуализации ткани и
            соединения её частей для получения одежды; \item  разработать и
                реализовать алгоритм визуализации плечевой одежды.
\end{itemize}

% В каждой из применымих областей в большей степени требуется реалистичость
% изображения нежели физическая точность[1]
% 
% Компьютерная графика, решающая задачу моделирования и визуализации явлений
% реального мира. Сремится повысить реалистичность виртуального мира.
% киноиндустрия, мультипликации, комьютерные игры, дизайн.  Везде требуется
% реалистичное представление одежды,
% 
% Поэтому добиваются в большей степени реалистичности изоброжения, нежели
% физической точности.[1] Также должны быть решены проблемы реалистичной
% визуализации складок на одежде и предотвращения взимопроникновения с твердыми
% телами и другими частями ткани, которые требуют больших вычислительных
% затрат.[7]
