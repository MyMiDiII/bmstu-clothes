\chapter*{Введение}
\addcontentsline{toc}{chapter}{Введение}

В современном мире компьютерная графика используется во многих ... Исследования
в этой области сосредоточены на моделироании и визуализации явлений реального
мира, но и придании ему реалистичности. Моделирование одежды и, как более
общего случая, ткани играют не последнюю роль в детализации виртуальных
сред.[4] Реалистичный вид одежды придает выразительности анимационным
персонажам в комьютерных играх и мультипликации [7]; в фильмах помогает сделать
неотличимыми реального человека, снятого на камеру, от, так называемого,
цифрового дублера -- виртуальной реалистичной копии, которая "выполняет"
сложные,  опасные для жизни трюки.[3] Также сегодя развивается идея виртуальной
примерочной в интернет магазинах[2]. Все это ... моделирование одежды
необходимым практически применимым.  Добавить про текстильную промышленность.

Тканные материалы обладают уникальными свойствами: гибкая, эластичная, легко
изменяет свою форму, хаотична (зависит от предыдущих положений -> выглядит
по-разному) Все это создает сложности при моделировании одежды. Полужило
поводом к разработки разнообразных методов, предназначенных для различных
целей.

Цель работы ~--- разработать программное обеспечение для реалистичной
визуализации плечевой одежды на примере футболки, предоставляющее возможность
изменения её положения (перемещение, вращение, масштабирование).

Для достижения поставленной цели необходимо решить следующие задачи:
\begin{itemize}[left=\parindent] \item  формально описать модель ткани, как
            части одежды; \item  проанализировать методы визуализации ткани и
            соединения её частей для получения одежды; \item  разработать и
                реализовать алгоритм визуализации плечевой одежды.
\end{itemize}

% В каждой из применымих областей в большей степени требуется реалистичость
% изображения нежели физическая точность[1]
% 
% Компьютерная графика, решающая задачу моделирования и визуализации явлений
% реального мира. Сремится повысить реалистичность виртуального мира.
% киноиндустрия, мультипликации, комьютерные игры, дизайн.  Везде требуется
% реалистичное представление одежды,
% 
% Поэтому добиваются в большей степени реалистичности изоброжения, нежели
% физической точности.[1] Также должны быть решены проблемы реалистичной
% визуализации складок на одежде и предотвращения взимопроникновения с твердыми
% телами и другими частями ткани, которые требуют больших вычислительных
% затрат.[7]
