\chapter{Технологическая часть}

\section{Средства реализации}

Для реализации программного обеспечения был выбран язык программирования
высокого уровня Python~\cite{python}, что обусловлено:
\begin{itemize}
    \item поддержкой объектно-ориентированного программирования;
    \item наличие необходимых библиотек для реализации поставленной задачи;
    \item доступностью и распространенностью учебной литературы и документации.
\end{itemize}

В качестве среды разработки выбран текстовый редактор Vim~\cite{vim} с
установленными плагинами автодополнения и поиска ошибок в процессе
написания. Данный редактор предоставляет возможности быстрого перемещения
по тексту программы и доступа к командной строке, что повышает скорость
разработки.

Для разработки интерфейса программного обеспечения использован Qt
Desingner~\cite{qt}, предоставляющий возможность автоматического преобразования
построенного интерфейса в код на необходимом языке.

Также в процессе разработки использовались следующие библиотеки:
\begin{itemize}
    \item PyQt5 для реализации работы с интерфейсом;
    \item numpy и ctypes --- для преобразования типов Python к типам, с которыми
        работает OpenGL;
    \item glm --- математическая библиотека для работы с векторами и матрицами.
\end{itemize}

\section{Реализация алгоритмов}

На листинге \ref{lst:01} представлена реализация алгоритма обновления
положений всех точных масс. На листинге \ref{lst:02} представлена
реализация алгоритма вычисления нового положения точечной массы. На
листинге \ref{lst:03} представлена реализация вычисления результирующей силы,
действующей на точечную массу. На листинге \ref{lst:04} представлена
реализация вычисления силы упругости от пружины.

\mylisting{Алгоритм обновления положений всех точечных
масс}{01}{133-135}{../../src/clothes/massspringsystem.py}

\mylisting{Алгоритм вычисления нового положения точечной
массы}{02}{94-105}{../../src/clothes/mass.py}

\mylisting{Алгоритм вычисления результирующей силы, действующей на точечную
массу}{03}{68-86}{../../src/clothes/mass.py}

\mylisting{Алгоритм вычисления силы упругости
пружины}{04}{70-74}{../../src/clothes/spring.py}

\section*{Вывод}

В данном разделе описаны средства реализации программного обеспечения, а также
представлены листинги реализованных алгоритмов.
