\chapter*{Введение}
\addcontentsline{toc}{chapter}{Введение}

Современные исследования в области компьютерной графики сосредоточены на
моделировании и визуализации явлений реального мира с максимальной
реалистичностью. Моделирование одежды и, как более общего случая, ткани играют
не последнюю роль в детализации виртуальных сред \cite{bib01}.
Реалистичный вид одежды придает выразительности анимационным персонажам в
комьютерных играх и мультипликации \cite{bib02}; в фильмах
помогает сделать неотличимыми реального человека, снятого на камеру, от, так
называемого, цифрового дублера --- виртуальной реалистичной копии, которая
"выполняет"\ сложные,  опасные для жизни трюки \cite{bib03}. Также
сегодня развивается идея виртуальной примерочной в интернет-магазинах
\cite{bib04}. Все это показывает практическую применимость
моделирования одежды, а следовательно, необходимость разработки методов её
визуализации.

Ткань, основа одежды, является материалом с уникальными свойствами: гибкостью и
изменением формы при небольшом воздействии \cite{bib05}. Они вносят в
рассматриваемые явления хаотичность, что  замечается в реальной жизни: каждый
раз, когда человек надевает тот или иной элемент одежды, многие детали выглядят
по-разному \cite{bib06}.  Перечисленные свойства усложняют задачу моделирования
тканных материалов по сравнению с моделированием твердых тел \cite{bib07}.
Стоит отметить также разные цели моделирования ткани. Так, в анимации акцент
делается на внешний вид конечного результата, в то время как в инженерном
сообществе, которое также работает с тканными материалами, ценится физическая
точность \cite{bib03}. Всё выше перечисленное приводит к тому, что существует
большое количество методов визуализации ткани, использующихся в каждой
конкретной ситуации. В данной курсовой работе ставится цель получения
изображения одежды и достижения его реалистичности.

Цель работы --- разработать программное обеспечение для реалистичной
визуализации плечевой одежды на примере футболки, предоставляющее возможность
изменения её положения (перемещение, вращение, масштабирование).

Для достижения поставленной цели необходимо решить следующие задачи:
\begin{itemize}[left=\parindent]
    \item  формально описать модель ткани, как части одежды;
    \item  проанализировать методы визуализации ткани и соединения её частей
           для получения одежды;
    \item  разработать и реализовать алгоритм визуализации плечевой одежды.
\end{itemize}

